%! Author = aybehrouz
%! Date = 2/26/21

% Preamble
\documentclass{article}

% Packages
%\usepackage{amsmath}
%\usepackage{amssymb}

% Document
\begin{document}
    \section{Consensus}\label{sec:consensus}

    \subsection{Estimating A User's Stake}\label{subsec:estimating-a-user's-stake}

    In a proof of stake system the influence of a user in the consensus protocol should be proportional to the amount
    of stake the user has in the system.
    Conventionally in these systems, for estimating a user's stake, we use the amount of native system tokens the
    user is holding.
    Unfortunately, one problem with this approach is that a strong attacker may be able to obtain a considerable
    amount of system tokens, for example by borrowing from a DEFI application, and use this stake to attack the system.

    To mitigate this problem, for calculating a user's stake, instead of using the raw ALGO balance, we use the
    minimum of a \emph{trust value} that the system has calculated for the user and the user's ALGO balance:
    \[
        Stake_{user} = \min (Balance_{user},Trust_{user})
    \]
    For estimating the value of $Trust_{user}$ we use the exponential moving average of the user's ALGO balance.
    Therefore, in our system a user who held ALGOs and participated in the consensus for a long time is more trusted
    than a new user with a higher balance.
    An attacker who has obtained a large amount of ALGOs, also needs to hold them for a long period of time before
    being able to attack our system.

    For calculating the exponential moving average of a time series at the time step $t$, we can use the following
    recursive formula:
    \[
        M_{t} = (1 - \alpha) M_{t - 1} + \alpha X_{t} = M_{t - 1} + \alpha (X_{t} - M_{t - 1})
    \]
    Where:
    \begin{itemize}
        \item The coefficient $\alpha$ is a constant smoothing factor between 0 and 1 which represents the degree of
        weighting decrease, A higher $\alpha$ discounts older observations faster.
        \item $X_{t}$ is the value of the time series at the time step $t$.
        \item $M_{t}$ is the value of the EMA at the time step $t$.
    \end{itemize}

    Usually an account balance will not change in every time step, and we can use older values of EMA for calculating
    $M_{t}$:
    \[
        M_{t} = (1 - \alpha)^{t - k}M_{k} + [1 - (1 - \alpha)^{t - k}]X
    \]
    Where:
    \[
        X = X_{k+1} = X_{k+2} = \dots = X_t
    \]
    When $|nx| \ll 1$ we can use the binomial approximation  $(1 + x)^n \approx 1 + nx$ to further simplify this
    formula:
    \[
        M_{t} = M_{k} + (t - k) \alpha (X - M_{k})
    \]

    For choosing the value of $\alpha$ we can consider the number of time steps that the trust value of a user needs
    for reaching a specified fraction of his account balance.
    We know that for large $n$ and $|x| < 1$ we have $(1 + x)^{n} \approx e^{nx}$, so by letting $k = 0$ and $t = n$
    we can write:
    \[
        \alpha =- \frac{\ln\left(1 - \frac{M_{n}}{X}\right)}{n}
    \]
    The value of $\alpha$ for a desired configuration can be calculated by this equation.
    For instance, we could calculate the $\alpha$ for a relatively good configuration in which $M_{n} = 0.8X$ and $n$
    equals to the number of time steps of 10 years.
\end{document}