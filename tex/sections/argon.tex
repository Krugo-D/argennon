%! Author = aybehrouz

\lstset{
    language=Java,
    morekeywords={main, initialize, constructor, external, virtual, map, account, signature, overrides, is, string},
    basicstyle=\small\sffamily,
    columns=fullflexible,
%   keywordstyle=\bfseries,
    breaklines=true,
    showstringspaces=false,
    mathescape
}

\section{Introduction}\label{sec:introduction2}

The Argon programming language is a class-based, object-oriented language designed for writing Argennon's smart
contracts. The Argon programming language is inspired by Solidity and is similar to Java, with a number of aspects
of them omitted and a few ideas from other languages included. Argon is designed to be fully compatible with
the Argennon Virtual Machine and be able to use all advanced features of the Argennon blockchain.

Argon Programs are organized as sets of packages. Each package has its own set of names for types, which helps
to prevent name conflicts. Every package can contain an arbitrary number of classes and a single contract. Every
Argon package corresponds to at max one AVM smart contract.


\section{Features Overview}\label{sec:features-overview}

\subsection{Static Classes}\label{subsec:static-classes}

In Argon a static class is an uninstantiable class which all its members and methods are static. Every static
class can define a special method \texttt{initialize} which can only be called in the \texttt{initialize}
method of the single contract of the package the static class is defined in.

A Static class is like a normal class which can only have one instance. So like normal classes, static classes can
be a super class or a subclass of other static classes and can implement interfaces.

\subsection{Contracts}\label{subsec:contract}

Every package of an Argon program can have one or zero contracts. A contact is a special static class which is
allowed to define methods with \texttt{external} visibility. When an Argon package is deployed as an AVM smart
contract, these external methods will define the contract's public interface.

\subsection{Access Level Modifiers}\label{subsec:access-level-modifiers}

Access level modifiers determine whether other classes can use a particular field or invoke a particular method or
if a method can be invoked externally by other smart contracts.

\begin{center}
    \begin{tabular}{llllll}
        \hline
        & Class & Package & Subclass & Program & World \\
        \hline
        private   & yes   & no      & no       & no      & no    \\
        protected & yes   & no      & yes      & no      & no    \\
        package   & yes   & yes     & yes      & no      & no    \\
        public    & yes   & yes     & yes      & yes     & no    \\
        external  & no    & no      & no       & yes     & yes   \\
        \hline
    \end{tabular}\label{tab:table}
\end{center}



\begin{lstlisting}[frame=TB, float, title=Argon contracts are static classes,label={lst:code1}]
// .
public class MirrorToken {
    private SimpleToken token;
    private SimpleToken reflection;

    // `initialize` is a special static method that is called by the AVM after the code of a contract
    // is stored in the AVM code area.
    public static void initialize(double supply1, double supply2) {
        // `new` does not create a new smart contract. It just makes an ordinary object.
        token = new SimpleToken(supply1);
        reflection = new SimpleToken(supply2);
    }
    // `main` is the only method of the application (i.e. smart contract) that can be called by other
    // applications. Every application should have exactly one main method defined in a class.
    // Alternativly the keyword `dispatcher` could be used instead of `main`.
    public static RestResponse main(RestRequest arg) {
        if (arg.pathMatches("\balances\{user}")) {
            account sender = arg.getParameter<account>("user");
            if (arg.operationIsPUT()) {
                verify(arg.toMessage(), sender, arg.getParameter<signature>("sig"));
                token.transfer(sender, recipient, amount);
                reflection.transfer(recipient, sender, Math.sqrt(amount));
            } else if (arg.operationIsGET()) {
            } else {
            }
        }
        arg.operationIsGET();
        arg.operationIsPUT();
        arg.getParameter<address>("user")
    }

    external void transfer(account sender, account recipient,
        double amount, signature sig) {
        // Checks if the signature is issued by the sender and verifies the calling of this function
        // with the current parameters.
        sig.verifyCallBy(sender);
        token.transfer(sender, recipient, amount);
        reflection.transfer(recipient, sender, Math.sqrt(amount));
    }

    external double balanceOf(account user) { return token.balanceOf(user); }

    external double balanceOfReflection(account user) { return reflection.balanceOf(user); }
}


package class SimpleToken {
    private map(account -> double) balances;

    // The visibility of a member without an access modifier will be the same as its defining
    // class so `SimpleToken` constructor has package visibility.
    constructor(double initialSupply) {
        // initializes the object
    }

    void transfer(account sender, account recipient, double amount) {
        if (balances[sender] < amount) throw("Not enough balance.");
        // implements the required logic...
    }
    // implements other methods...
}
\end{lstlisting}

\subsection{Shadowing}\label{subsec:shadowing}

If a declaration of a type (such as a member variable or a parameter name) in a particular scope (such as an
inner block or a method definition) has the same name as another declaration in the enclosing scope, it will
result in a compiler error. In other words, the Argon programming language does not allow shadowing.
