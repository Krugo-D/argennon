%! Author = aybehrouz

The Argennon blockchain is a sequence of blocks. Every block represents an ordered list of transactions, intended to be
executed by the Argennon Virtual Machine (AVM). The first block of the blockchain, the \emph{genesis} block, is a spacial
block that fully describes the initial state of the AVM. Every block of the Argennon blockchain thus corresponds to a
unique AVM state which can be calculated deterministically from the genesis block.

A block of the Argennon blockchain contains the following information:

\begin{center}
    \begin{tabular}{||c||}
        \hline
        \textbf{Block} \\ [0.5ex]
        \hline\hline
        commitment to the staking database            \\ [1.2ex]
        commitment to the method database             \\ [1.2ex]
        commitment to the heap database               \\ [1.2ex]
        commitment to the set of transactions         \\ [1.2ex]
        a consecutive list of block certificates issued     \\
        by validators' committee (if any)           \\ [1.2ex]
        clustering directives                         \\ [1.2ex]
        random seed                                   \\ [1.2ex]
        previous block hash                           \\ [1.2ex]
        \hline
    \end{tabular}
\end{center}

\subsection{Block Validation}\label{subsec:block-validation}

Having the previous AVM state, the transaction list and the clustering directives of a block, a node can calculate
commitments to the staking, method and heap databases of the current block by emulating the AVM execution. If the
node can obtain the previous block commitments from a trusted source, it does not need to have a trusted local
copy of the AVM state,
and it can reliably retrieve the required storage pages from a ZK-EDB server. We call this type of block verification
\emph{conditional} block validation, since the validity of the current block is
conditioned on the validity of the previous block.

Interestingly, conditional block validation of multiple blocks can be done in parallel. If a node has enough bandwidth
and computational resources, it can conditionally verify any number of blocks from a previously created blockchain
simultaneously and in parallel. As we will see in Section~\ref{subsec:validators-committee}, this property plays an
important role in the Argennon consensus protocol.

To some extent, conditional validation of a single block could be parallelized as well. Many transactions
in a block are actually independent and the order of their execution does not
matter. These transactions can be safely validated in parallel. Section~\ref{sec:concurrency} further
develops this concept.


\subsection{Block Certificate}\label{subsec:block-certificate}

An Argennon block certificate is an aggregate signature of some predefined subset of accounts. This predefined subset
is called the certificate committee and their signature ensures that the certified block is conditionally
valid given the validity of some previous block.

Argennon uses BLS aggregate signatures to represent block certificates. To better understand block certificates and
the Argennon consensus protocol, we need to briefly review the BLS signature scheme and its aggregation mechanism.

The BLS signature scheme operates in a prime order group and supports simple threshold signature generation,
threshold key generation, and signature aggregation. To review, the scheme uses the following ingredients:

\newcommand{\G}{\mathbb{G}}
\newcommand{\Z}{\mathbb{Z}}
\newcommand{\adv}{{\cal A}}
\newcommand{\bdv}{{\cal B}}
\newcommand{\deq}{\mathrel{\mathop:}=}
\newcommand{\SK}{\mathit{sk}}
\newcommand{\PK}{\mathit{pk}}
\newcommand{\C}{\mathit{cert}}
\newcommand{\APK}{\mathit{apk}}
\newcommand{\DPK}{\mathit{\Delta pk}}
\newcommand{\MM}{\mathcal{M}}
\newcommand{\xwedge}{\, \operatorname{\text{$\wedge$}}\, }
\newcommand{\abs}[1]{\lvert #1 \rvert}
\newcommand{\Hm}{H_0}
\newcommand{\Hpk}{H_1}
\newcommand{\qHpk}{Q_{\Hpk}}
\newcommand{\qHm}{Q_{\Hm}}
\newcommand{\qsig}{Q_{\text{sig}}}

\begin{itemize}
    \item An efficiently computable \emph{non-degenerate} pairing $e:\G_0 \times \G_1 \to \G_T$
    in groups $\G_0$, $\G_1$ and $\G_T$ of prime order $q$. We let $g_0$ and $g_1$ be generators
    of $\G_0$ and $\G_1$ respectively.
    \item A hash function $H_0: \mathcal{M} \rightarrow \mathbb{G}_0$, where $\mathcal{M}$ is the message space.
    The hash function will be treated as a random oracle.
\end{itemize}

The BLS signature scheme is defined as follows:

\begin{itemize}
    \item $\textbf{KeyGen}()$: choose a random $\alpha$ from $\Z_q$ and set $h \gets g_1^\alpha \in \G_1$.
    output $\PK \deq (h)$ and $\SK \deq (\alpha)$.
    \item $\textbf{Sign}(\SK, m)$: output $\sigma \gets \Hm(m)^\alpha \in \G_0$.
    The signature $\sigma$ is a \emph{single} group element.
    \item $\textbf{Verify}(\PK,m,\sigma)$: if $e(g_1, \sigma) = e\big(\PK,\ \Hm(m)\big)$  then output "accept",
    otherwise output "reject".
\end{itemize}

Given triples $(\PK_i,\ m_i,\ \sigma_i)$ for $i=1,\ldots,n$,
anyone can aggregate the signatures $\sigma_1,\ldots,\sigma_n \in \G_0$
into a short convincing aggregate signature $\sigma$ by computing
\begin{equation}
    \label{eq:agg}
    \sigma \gets \sigma_1 \cdots \sigma_n \in \G_0\ .
\end{equation}
Verifying an aggregate signature $\sigma \in \G_0$ is done by checking that
\begin{equation}
    \label{eq:aggdiff}
    e(g_1, \sigma) = e\big(\PK_1,\ \Hm(m_1)\big) \cdots e\big(\PK_n,\ \Hm(m_n)\big)\ .
\end{equation}
When all the messages are the same ($m = m_1 = \ldots = m_n$), the verification relation~\eqref{eq:aggdiff} reduces to
a simpler test that requires only two pairings:
\begin{equation}
    \label{eq:aggsame}
    e(g_1, \sigma) = e\Big(\PK_1 \cdots \PK_n,\ \Hm(m)\Big)\ .
\end{equation}
We call $\APK=\PK_1 \cdots \PK_n$ the aggregate public key.

To defend against \emph{rogue public key} attacks, Argennon uses Prove Knowledge of the Secret Key (KOSK) scheme. As we
explained in Section~\ref{sec:accounts}, when an account is created its public keys need to be registered in
the ARG smart contract. Therefore, the KOSK scheme can be easily implemented in Argennon.

Because it is not usually possible to collect the signatures of all members of a certificate committee, an Argennon
block certificate essentially is an Accountable-Subgroup Multi-signature (ASM). Argennon uses a simple ASM scheme
based on BLS aggregate signatures.

Argennon block certificates constitute an ordered sequence based on the order of blocks they certify. If we show
the $i$-th certificate\footnote{note that the $i$-th certificate is not
necessarily the certificate of the $i$-th block.} of committee $C$ with $\C_i$, and the set of signers
with $S_i$, then the block certificate $\C_i$ can be considered as a tuple:
\begin{equation}
    \C_i=(\sigma_i,\ C-S_{i})\label{eq:cert}\ ,
\end{equation}
where $\sigma_i$ is the aggregate signature issued by $S_i$.

The aggregate public key of the certificate can
be calculated from:
\begin{equation}
    \APK_i=\APK_C\APK_{C-S_i}^{-1}\label{eq:aggCertPK}\ ,
\end{equation}
where $\APK_{A}$ shows the aggregate public key of all accounts in $A$.

Alternately, we can use $\APK_{i-1}$ to calculate the aggregate public key:
\begin{equation}
    \APK_i=\APK_{i-1}\APK_{S_i-S_{i-1}}\APK_{S_{i-1}-S_i}^{-1}\ .\label{eq:aggPK-2}
\end{equation}

When an Argennon account is created, both its $\PK$ and $\PK^{-1}$ is registered in the ARG smart contract, so the
inverse of any aggregate public key can be easily computed.\footnote{since the group operator of a cyclic
group is commutative, we have $(ab)^{-1}=a^{-1}b^{-1}$.}

