The most conventional use of blockchains is in financial applications. This gives a crucial importance to the
security of the consensus protocol used in a blockchain. Unfortunately, many currently used blockchains are
vulnerable to a
certain type of consensus attack, known as the bribery attack. In a bribery attack, an adversary tries to corrupt
participants of a protocol and make them violate the protocol by offering them money.

At the time of writing of this document,
the total mining reward for a bitcoin block is around \$150,000. If we assume that mining rewards accurately define
the utility function of a bitcoin miner, one could hire all hashing power of the bitcoin network for one hour by
spending
only \$750,000. The situation is not much different for PoS blockchains, as long as the total stake\footnote{Here, by
stake, we mean a real number measuring the total interest of a user in the system. We are not referring, in
particular, to the locked amount of a user's money that is known as stake in some PoS protocols.}
of the validator set of a block is a relatively small value. This is more severe in blockchains that use
randomly selected small sets of validators. These small sets could be easily bribed and selecting these random sets
by hidden random procedures would not help, since the validator himself knows he has been selected, before casting
his vote.

It appears that the only solution to this important vulnerability is to effectively participate all the stake of the
system in the consensus protocol. This large participation makes the protocol resilient against bribery or
collusion because the adversary would have to spend unrealistic amount of money to bribe enough users.

However, for effective participation in the consensus protocol, a validator needs to be able to detect
illegal transactions. Detecting illegal transaction can be done by accessing the ledger
state and executing transactions according to the protocol rules. The
ledger state, even for small blockchains, could be several
hundreds of gigabytes, and executing transactions could easily become costly when a blockchain is acting as a
smart contract platform. This computational and storage overhead, in practice, could prevent most of
ordinary users from any type of participation in the process of securing a blockchain.

Although a fully decentralized blockchain based on the participation of every user looks appealing, it is not as perfect
as it might seem. A blockchain consensus protocol relies on a network of computers, not humans. Ordinary users use
simple and similar computer systems. That means, they all have similar vulnerabilities which could be used by an
adversary to catastrophically attack the consensus protocol. For example, a malware using a common zero-day
vulnerability might be able to infect a large portion of normal personal computers. This malware could be used by an
adversary to control the majority of participant computers in the consensus protocol and compromise the security of the
blockchain.

Securing a computer system against cyberattacks needs planning ahead and access to engineering resources.
Special software and hardware, like custom-built operating systems and isolated specialized hardware is required.
This is not something a normal user can afford. Only powerful centralized entities having large financial and
technical resources, could build
systems that are resilient against sophisticated cyberattacks. In this regard, we have to admit, a centralized system is
arguably superior to a decentralized\footnote{Note that decentralized and distributed are two different concepts.}
system.

To overcome these difficulties, Argennon uses a hybrid Proof of Stake consensus protocol. A small committee of
delegates is democratically elected
by the Argennon users and is responsible for minting new blocks of the blockchain. Each
minted block is required to get approved by its corresponding validator assembly. Assemblies are very large sets of
validators including at least three percent of the total stake of the system. A block is approved if it gets
approval votes from at least \(2/3\) of the total stake of its validator assembly.

Each block of the Argennon blockchain contains a set of Computational Integrity (CI) statements and a commitment to
the corresponding ledger state of the block. The CI set defines how
the ledger state of the block can be reached from the state of its previous block via a set of intermediate states.
Each individual CI statement defines an ordered list of external requests\footnote{External requests in Argennon are
similar to transactions in older blockchains.} and determines the intermediate states before and after executing those
requests:
\[
    \tau_{(s_i,s_{i+1},\mathbb{R})} \coloneqq \text{``$s_{i+1}$ is the next state after executing external requests
        $\mathbb{R}$ on state $s_i$''}
\]
Validators can\footnote{Using the Argennon cloud is optional for a validator.} receive succinct cryptographic proofs
of these CI statements from the Argennon cloud. They can also receive the state of the previous
block (the required part of it) and proofs that show the received data is consistent with the previous block commitment.

Verifying a succinct proof can be exponentially faster than replaying the computation. Moreover, verifications of
different CI statements are independent and can be done in parallel. As a result, a validator can use multiple cores for
verifying CI statements of a single block and different validator assemblies can simultaneously and independently
verify different blocks.
In addition, proof generation of different CI statements similarly can be done in parallel. However, for parallel
proof generation, the state transition needs to be known in advance. That's why in Argennon, delegates do not try
to generate proofs and focus all their computational power on executing external requests and
generating the state transition as fast as possible.

%% Argennon execution layer is optimized for this ....

Independence of CI statements is useful, but is not enough for having a truly scalable blockchain. To increase
parallelism, the Argennon protocol enforces all external requests to pre-declare their memory access locations. That
would enable a block proposer\footnote{In Argennon, delegates are the only block proposers.} to use Data Dependency
Analysis\footnote{See Section~\ref{sec:concurrency}} (DDA) to indicate independent sets of external requests (i.e.\
transactions) and use those sets for parallel processing. More importantly, these sets can be used for generating CI
statements that
are defined on the same initial state and their proof can be generated independently without the need to calculate
the state transition in advance.

The Argennon protocol strongly incentivizes the formation of a permission-less network of Publicly Verifiable
Cloud (PVC) servers. To do so, the Argennon protocol conducts repetitive automatic lotteries between PVC servers.
A PVC server can increase its chance of winning by (i) generating proofs for more CI statements, (ii) providing the
state data to more validators and, (iii) storing all parts of the state instead of more frequently used parts.

A PVC server in Argennon, is a conventional data server which uses its computational and
storage resources to help the Argennon network process transactions. This encourages the development
of conventional networking, storage and compute hardware, which can benefit all areas of information technology.
