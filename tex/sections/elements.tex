%! Author = aybehrouz


\section{Applications}\label{sec:applications}

An Argennon application or smart contract is an HTTP server which is represented by an Argennon Standard
Representation (ASR) and whose state is stored in the Argennon blockchain. Each Argennon application is identified by
a unique application identifier.

An application identifier, \texttt{applicationID}, is
a unique prefix code generated by the \emph{applications} prefix tree. (See Section~\ref{sec:identifiers}.)
An application identifier can be considered as the address of an application and has the following standard symbolic
representation:
\begin{verbatim}
<application-id> ::= <decimal-prefix-code>
<decimal-prefix-code> ::= <dec-num>"."<decimal-prefix-code> | <dec-num>
\end{verbatim}
where \texttt{<dec-num>} is a normal decimal number between $0$ and $255$. For example \texttt{21.255.37},
\texttt{0}, \texttt{11.6} and \texttt{2.0.0.0.0}, are valid application addresses.

Argennon has two special smart contracts: the \emph{root smart contract}, also called the \emph{root application}, and
the \emph{ARG smart contract}, which is also called the \emph{Argennon smart contract} or the \emph{ARG application}.

Argennon application use HTTP as the application protocol and they are advised to have a RESTful API design.

\subsection{The Root Application}\label{subsec:the-root-app}

The root application or the root smart contract, with \texttt{applicationID = 0}, is a privileged smart contract
responsible for installation/uninstallation of other smart contracts. The Argennon's root smart contract
performs three main operations:

\begin{itemize}
    \item Installation of new Argennon applications and determining the update policy of a smart
    contract: if the contract is updatable or not, which accounts or smart contracts can update or uninstall
    the contract, and so on.
    \item Removing an Argennon application (if allowed).
    \item Updating an Argennon application (if allowed).
\end{itemize}

The root smart contract is a mutable smart contract and can be updated by the Argennon governance system.
(See Section~\ref{sec:adags})

\subsection{The ARG Application}\label{subsec:the-arg-app}

The ARG application or the ARG smart contract,
with \texttt{applicationID = 1}, controls the ARG token, the main
currency of the Argennon blockchain. This smart contract also manages a database of public keys and
handles signature verification.

The ARG smart contract is a mutable smart contract and can be updated by the Argennon governance system.


\section{Accounts}\label{sec:accounts}

Argennon accounts are entities defined inside the ARG application.
Every Argennon account is uniquely identified by a prefix code generated using \emph{accounts} prefix
tree. (See Section~\ref{sec:identifiers}) An account
identifier can be considered as the address of an account and has the following standard symbolic representation:
\begin{verbatim}
<account-id> ::= "0x"<hex-num>
\end{verbatim}
where \texttt{<hex-num>} is a hexadecimal number, using lower case
letters \texttt{[a-f]} for showing digits greater than $9$.

For example \texttt{0x24ffda}, \texttt{0x0} and \texttt{0x03a0000}, are valid standard symbolic
representations of account addresses.

A new account can be created by sending a proper HTTP request to the ARG smart contract. For creating
a new account two public keys need to be provided by the caller and registered in the Argennon smart contract.
One public key will be used for issuing digital signatures, and the other one will be used for voting. The
provided public keys need to meet certain cryptographic requirements,\footnote{Argennon uses Prove
Knowledge of the Secret Key (KOSK) scheme.} and can not be already registered in the system.

If the owner of the new account is an application, the \texttt{applicationID} of the owner will be registered in the
ARG smart contract and no public keys are needed. An application can own an arbitrary number of accounts.

\note{Explicit key registration enables Argennon to decouple cryptography from the blockchain design. In this way,
    if the cryptographic algorithms used become insecure for some reason, for example because
    of the introduction of quantum computers, they could be easily upgraded.}


\section{Transactions}\label{sec:transactions}

An Argennon transaction consist of an HTTP request made by a user to an Argennon application, a resource
declaration object and a list of signed messages. Transactions can only be
issued by users and applications can not create transactions. An Argennon transaction is also called
an \emph{external request}.

\subsection{Resource Declaration}\label{subsec:resource-declaration}

Every Argennon transaction is required to provide the following information as an upper bound for the
resources it needs:

\begin{itemize}
    \item Maximum AscEE clocks
    \item The list of applications the request will call
    \item The list of access blocks the request needs
    \item \texttt{maxSize} for chunks it wants to expand
    \item \texttt{minSize} for chunks it wants to shrink
    \item A list of applications it will update (if any)
\end{itemize}

If a transaction tries to violate any of these predefined limitations, it will be considered failed, and the network
can receive the proposed fee of that transaction.

\begin{lstlisting}[language=python, frame=TB, float, title=An Argennon transaction in YAML format,
    label={lst:txn-example}]
---
request: |
    PATCH /balances/0x95ab HTTP/1.1
    Content-Type: application/json; charset=utf-8
    Content-Length: 46

    {"to":0xaabc,"amount":1399,"sig":0}

messages:
   - issuer: 0x95ab000000000000
     msg: {"to":0xaabc000000000000,"amount":1399,"forApp":0x100000000000000,"nonce":11}
     sig: LNUC49Lhyz702uszzNcfaU3BhPIbdaSgzqDUKzbJzLPTlFS2J9GzHl-cDKb

caps:
    maxClocks: 150 # maximum number of AVM execution clocks
    apps: [1,124.16]
    read: [(2654,3),(15642,0),(15642,1),(15642,3)]
    write: [(15642,0),(20154,0),(20154,1)]
\end{lstlisting}


