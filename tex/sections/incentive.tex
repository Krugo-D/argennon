%! Author = aybehrouz


\subsection{Certificate Rewards}\label{subsec:rewards}

When the certificate of a committee of validators is included in a block, the signers of that certificate and the
delegates will be rewarded. Every validator will be rewarded equally. The delegates will be rewarded
proportional to the total stake of the signers of the certificate and their reward increases if the
certificate has more signers.

Rewards will not be distributed instantly, instead they will be distributed at the end of the staking period.
This will facilitate efficient implementations which avoid frequent updates in the AVM storage.
Rewards of every staking period depends on the amount of fees that are collected during that period.

\subsection{Penalties}\label{subsec:penalties}

If an account behaves maliciously, and that behaviour could not have happened due to a mistake, by providing a proof
in a block, the account will be disabled forever in the ARG smart contract. Disabling an account in the
ARG smart contract will prevent that account from signing any valid signature in the future.

Only behaviours will be punished that can not happen due to a mistake or an attack. These behaviours include:
\begin{itemize}
    \item Signing a certificate for a block that is not conditionally valid.
    \item Signing a certificate for two different blocks at the same height if none of them
    is a fork block or a seal block.\footnote{Signing
    a fork block and a normal block at the same height usually is a malicious behaviour. However, it will not be
    penalized because there are circumstances that a honest user could mistakenly do that.}
\end{itemize}

\subsection{Incentives for ZK-EDB Servers}\label{subsec:zk-edb-servers}

The incentive mechanism for ZK-EDB servers should have the following properties:

\begin{itemize}
    \item It incentivizes storing all storage pages and not only those pages that are used more frequently.
    \item It incentivizes ZK-EDB servers to actively provide the required storage pages for validators.
    \item Making more accounts will not provide any advantage for a ZK-EDB server.
\end{itemize}

For our incentive mechanism, we require that every time a validator receives a storage page from a ZK-EDB, after
validating the data, he give a receipt to the ZK-EDB server. In this receipt the validator signs the
following information:

\begin{itemize}
    \item \texttt{ownerAddr}: the account address of the ZK-EDB server.
    \item \texttt{receivedPageID}: the ID of the received page.
    \item \texttt{round}: the current block number.
\end{itemize}

\note{In a round, an honest validator never gives a receipt for an identical page to two different ZK-EDB servers.}

To incentivize ZK-EDB servers, a lottery will be held every round,\footnote{A round is the time interval between
two consecutive blocks.} and a predefined amount of ARGs from
\texttt{dbFeeSink} account will be distributed between winners as a prize. This prize will be divided equally
between all \emph{winning tickets} of the lottery.

\note{One ZK-EDB server could own multiple winning tickets in a round.}

To run this lottery, every round, based on the current block seed, a collection of \emph{valid} receipts will be
selected randomly as the \emph{winning receipts} of the round. A receipt is \emph{valid} in round $r$ if:

\begin{itemize}
    \item The signer was a member of the validators' committee of the block $r - 1$ and signed the block certificate.
    \item The page in the receipt was needed for validating the \textbf{previous} block.
    \item The receipt round number is $r - 1$.
    \item The signer did not sign a receipt for the same storage page for two different ZK-EDB servers in
    the previous round.
\end{itemize}
For selecting the winning receipts we could use a random generator:
\begin{verbatim}
IF random(seed|validatorPK|receivedPageID) < winProbability THEN
    the receipt issued by validatorPK for receivedPageID is a winner
\end{verbatim}
\begin{itemize}
    \item \texttt{random()} produces uniform random numbers between 0 and 1, using its input argument as a seed.
    \item \texttt{validatorPK} is the public key of the signer of the receipt.
    \item \texttt{receivedPageID} is the ID of the storage page that the receipt was issued for.
    \item \texttt{winProbability} is the probability of winning in every round.
    \item \texttt{seed} is the current block seed.
    \item \texttt{|} is a concatenation operator.
\end{itemize}

Also, based on the current block seed, a random storage page is
selected as the challenge of the round. A ZK-EDB server that owns a winning receipt needs to broadcast a \emph{winning
ticket} to claim his prize. The winning ticket consists of a winning receipt and a \emph{solution} to the round
challenge. Solving a round challenge requires the content of the storage page which was selected as the round
challenge. This will encourage ZK-EDB servers to store all storage pages.

A possible choice for the challenge solution could be the cryptographic hash of the content of the challenge
page combined with the server account address:

\texttt{hash(challenge.content|ownerAddr)}

The winning tickets of the lottery of round $r$ need to be included in the block of the round $r$,
otherwise they will be considered expired. Validation and prize distribution for the winning tickets of round
$r$ will be done in the round $r + 1$. This way, \textbf{the content of the challenge page could be
kept secret during the lottery round.} Every winning ticket will get an equal share of the lottery prize.

\subsection{Memory Allocation and De-allocation Fee}\label{subsec:memory-allocation-and-de-allocation}

Every $k$ round the protocol chooses a price per byte for AVM memory. When a smart contract executes a heap
allocation instruction, the protocol will automatically deduce the cost of the allocated memory from the ARG
address of the smart contract.

To determine the price of AVM memory, Every $k$ round, the protocol calculates \texttt{dbFee} and
\texttt{memTraffic} values. \texttt{dbFee} is the aggregate amount of collected database fees, and
\texttt{memTraffic} is the total memory traffic of the system. For calculating the memory traffic of the system
the protocol considers the total size of all the memory pages that were accessed for either reading or writing
during a time period. These two values will be calculated for the last $k$ rounds and the price per byte of
AVM memory will be a linear function of \texttt{dbFee/memTraffic}

When a smart contract executes a heap de-allocation instruction, the protocol will refund the cost of
de-allocated memory to the smart contract. Here, the current price of AVM memory does not matter and the protocol
calculates the refunded amount based on the average price the smart contract had paid for that allocated memory.
This will prevent smart contracts from profit taking by trading memory with the protocol.