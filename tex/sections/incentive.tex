%! Author = aybehrouz

\subsection{Fees}\label{subsec:fees}

The Argennon protocol does not explicitly define any fees for normal transactions. Only for high priority
transactions a fixed fee is determined by the governance system (See Section ...). Because the protection of the
Argennon network against spams and DOS attacks is mostly done by the delegates, they are also responsible for
determining and collecting transaction fees. A good fee collection policy could considerably increase the chance of
delegates for being reelected in the next terms, therefore they are incentivized to use creative and effective
methods\footnote{For example, they may
allow a limited number of free transactions per month for every account.}.

In Argennon fee payment can be done off-chain or on-chain. Off-chain fee payment is more efficient and flexible but
requires some level of trust in the delegates. For trust-less fee payment, the Argennon protocol provides the
concept of request attachments (See Section~\ref{sec:attachments}).
When a user does not want to use off-chain fee payment methods, he can simply define his transaction as the attachment
of the fee payment transaction. That way, the fee payment transaction will be performed only if the attached
transaction is also included in the same block.

While transaction fee is not enforced by the Argennon protocol, there are other types of fee that are mandatory: the
\emph{database fee} and the \emph{block fee}. Both of these fees are required to be paid for every block of the Argennon
blockchain and are paid by the delegates. The block fee is a constant fee that is paid for each new block of the
blockchain and its amount is
determined by the ADAGs. The database fee depends on the data access and storage overhead that a new block
is imposing on the Argennon storage cloud. The amount of this fee is determined by the ADAGs, and is collected in
a special account: the \texttt{dbFeeSink}.

\subsection{Certificate Rewards}\label{subsec:rewards}

The validators who sign the certificate of a block will receive the block fee paid for that block. Every validator
will be rewarded
proportional to his stake (i.e voting power). As we mentioned before the block fee is a constant fee which the
delegates pay for each block.

Rewards will not be distributed instantly, instead they will be distributed at the end of the staking period.
This will facilitate efficient implementations which avoid frequent updates in the Argennon storage.

As long as ARG is allowed to be minted and its cap is not reached, the delegates will receive a reward at the
\textbf{end} of their election term. This reward will consist of newly minted ARGs, and its amount will be
determined by the ADAGs. In addition, for each block certificate that is added to the Argennon blockchain some amount
of ARGs will be minted and added to the \texttt{dbFeeSink} account.

\subsection{Penalties}\label{subsec:penalties}

If an account behaves maliciously, and that behaviour could not have happened due to a mistake, by providing a proof
in a block, the account will be disabled forever in the ARG smart contract. Disabling an account in the
ARG smart contract will prevent that account from signing any valid signatures in the future.

Only behaviours will be punished that can not happen due to a mistake or an attack. These behaviours include:
\begin{itemize}
    \item Signing a certificate for a block that is not conditionally valid.
    \item Signing a certificate for two different blocks at the same height if none of them
    is a fork block or a seal block.\footnote{Signing
    a fork block and a normal block at the same height usually is a malicious behaviour. However, it will not be
    penalized because there are circumstances that an honest user could mistakenly do that.}
\end{itemize}

\subsection{Incentives for PV-DB Servers}\label{subsec:PV-DB-servers}

The incentive mechanism for PV-DB servers should have the following properties:

\begin{itemize}
    \item It incentivizes storing all storage pages and not only those pages that are used more frequently.
    \item It incentivizes PV-DB servers to actively provide the required storage pages for validators.
    \item Making more accounts will not provide any advantage for a PV-DB server.
\end{itemize}

For our incentive mechanism, we require that every time a validator receives a storage page from a PV-DB, after
validating the data, he gives a receipt to the PV-DB server. In this receipt the validator signs the
following information:

\begin{itemize}
    \item \texttt{ownerAddr}: the account address of the PV-DB server.
    \item \texttt{receivedPageID}: the ID of the received page.
    \item \texttt{round}: the current block number.
\end{itemize}

\note{In a round, an honest validator never gives a receipt for an identical page to two different PV-DB servers.}

To incentivize PV-DB servers, a lottery will be held every round,\footnote{A round is the time interval between
two consecutive blocks.} and a predefined amount of ARGs from
\texttt{dbFeeSink} account will be distributed between the winners as a prize. This prize will be divided equally
between all \emph{winning tickets} of the lottery.

\note{One PV-DB server could own multiple winning tickets in a round.}

To run this lottery, every round, based on the current block seed, a collection of \emph{valid} receipts will be
selected randomly as the \emph{winning receipts} of the round. A receipt is \emph{valid} in round $r$ if:

\begin{itemize}
    \item The signer was a member of the validators' committee of the block $r - 1$ and signed the block certificate.
    \item The page in the receipt was needed for validating the \textbf{previous} block.
    \item The receipt round number is $r - 1$.
    \item The signer did not sign a receipt for the same storage page for two different PV-DB servers in
    the previous round.
\end{itemize}
For selecting the winning receipts we could use a random generator:
\begin{verbatim}
IF random(seed|validatorPK|receivedPageID) < winProbability THEN
    the receipt issued by validatorPK for receivedPageID is a winner
\end{verbatim}
\begin{itemize}
    \item \texttt{random()} produces uniform random numbers between 0 and 1, using its input argument as a seed.
    \item \texttt{validatorPK} is the public key of the signer of the receipt.
    \item \texttt{receivedPageID} is the ID of the storage page that the receipt was issued for.
    \item \texttt{winProbability} is the probability of winning in every round.
    \item \texttt{seed} is the current block seed.
    \item \texttt{|} is the concatenation operator.
\end{itemize}

Also, based on the current block seed, a random storage page is
selected as the challenge of the round. A PV-DB server that owns a winning receipt needs to broadcast a \emph{winning
ticket} to claim his prize. The winning ticket consists of a winning receipt and a \emph{solution} to the round
challenge. Solving a round challenge requires the content of the storage page which was selected as the round
challenge. This will encourage PV-DB servers to store all storage pages.

A possible choice for the challenge solution could be the cryptographic hash of the content of the challenge
page combined with the server account address:

\texttt{hash(challenge.content|ownerAddr)}

The winning tickets of the lottery of round $r$ need to be included in the block of the round $r$,
otherwise they will be considered expired. However, finalizing and prize distribution for the winning tickets
should be done in a later round. This way, \textbf{the content of the challenge page could be
kept secret during the lottery round.} Every winning ticket will get an equal share of the lottery prize.
