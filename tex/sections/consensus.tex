%! Author = aybehrouz

There is two primary types of certificate committees in Argennon: the committee of \emph{delegates} and the committee
of \emph{validators}. Argennon has one committee of delegates and $m$ committees of validators.

The committee of delegates generates a certificate for every block of the Argennon blockchain, and each
committee of validators generates a certificate every $m$ blocks. A validators' committee will
certify a block only if it has already been certified by the committee of delegates and every committee
generates a certificate for a different block. A block is considered
final only after it gets a certificate from both of these committees.

In addition to primary committees, Argennon has several community driven committees. Certificates of these
committees are not required for block finality, but they could be used by the members of the
validators' committee to better decide about the validity of a block.

When an anomaly is detected in the consensus mechanism, the \emph{recovery} protocol is initiated by nodes. The
recovery protocol is designed to be resilient to many types of attacks in order to be able to restore the normal
functionality of the system.

\subsection{The Committee of Delegates}\label{subsec:the-committee-of-delegates}

The committee of delegates is a small committee of trusted delegates, elected by Argennon users either through the
Argennon Decentralized Autonomous Governance system (ADAGs\footnote{pronounced \textipa{/eI-dagz/}.}) or by emergency
voting during the recovery protocol. At the start of the Argennon mainnet, this committee will have
five members, and later its size could be changed by the ADAGs in a procedure described
in Section~\ref{ch:governance}.

The committee of delegates is responsible for creating new blocks of the Argennon blockchain, and it issues a
certificate for every block of the Argennon blockchain. A certificate needs to be signed
by \textbf{all} of the committee members in order to be considered valid.

Usually, the delegates are large organizations, and they have enough computational resources to generate blocks
very fast. However, a block is not completely final if it does not have the certificate of the validators' committee.
A certified block by the delegates will not be accepted by the network, if the last block certified by
the validators is behind it more than a certain number of blocks.

The committee of delegates may use any type of agreement protocol to reach consensus on the
next block. Usually a very simple and fast protocol can do the job: one of the members
is randomly chosen as the proposer, and other members vote "yes" or "no" on the proposed block.

If one of the delegates lose its network connectivity, no new blocks can be generated. For this reason,
the delegates should invest on different types of communication infrastructure, to make sure they never lose
connectivity to each other and to the Argennon network.

\subsection{The Committee of Validators}\label{subsec:validators-committee}

The Argennon protocol calculates a stake value for every account, which is an estimate of a user's stake in the
system. This value is measured in ARGs. Every $n$ block randomly $m$ committees are selected from
accounts whose stake value is higher than \texttt{minValidatorsStake} threshold. This
threshold is determined by the ADAGs, but it can never be higher than $100$ ARGs.
These committees are chosen in a
way that the total stake of members of each committee is approximately equal and every account is a member of at
least one committee.

Every member of a validators committee has a status which can be either \texttt{online} or \texttt{offline}.
This status is stored in the ARG smart contract, and it can be changed through a method invocation
from this smart contract. When an account sets its status to \texttt{offline}, it receives a small reward, and
it can not change it back to \texttt{online} for \texttt{statusCoolDown} number of blocks.

\note{When an account changes its status, the change has no effect until the block containing the status
change transaction gets certified by the committee of validators.}

A block certificate issued by some members of a validators committee is considered valid, if according to
the staking database of the previous \textbf{block certified by the same committee}, we have:
\begin{itemize}
    \item The total stake of \texttt{online} members is higher than \texttt{minOnlineStake} percent of the
    total stake of all members. This threshold can be changed by the ADAGs, but it can never be lower
    than 70 percent.
    \item All signers of the certificate have \texttt{online} status.
    \item The sum of stake values of the certificate signers is higher than 80 percent of the total stake
    of committee members that have \texttt{online} status.
\end{itemize}

If the total online stake of a committee goes lower than \texttt{minOnlineStake} threshold, nodes initiate the recovery
consensus protocol.

As we mentioned before, every $m$ blocks of the Argennon blockchain needs a certificate from a specific committee
of validators. To decide about signing the block certificate, an honest member of the validators'
committee checks the conditional validity of the committee block, and if the block is valid he issues an "accept"
signature, otherwise he issues a "reject" signature.

The value of $m$ is determined by the ADAGs. but it can never be higher than $25$. This way, it is guaranteed
that on average, any block of the Argennon blockchain is validated by at least $0.02$ of the total ARG supply.

If a committee of validators, at any point issues a "reject" certificate for a block, the network will initiate the
recovery protocol.

\subsubsection{Signature Aggregation}

In Argennon, signature aggregation is mostly performed by ZK-EDB servers. To distribute the aggregation workload
between different servers, The committee of validators is divided into pre-determined groups, and each ZK-EDB
server is responsible for signature aggregation of one group. To make sure that there is enough redundancy, the number
of groups should be less than the number of ZK-EDB servers and each group should be assigned to
multiple ZK-EDB servers.

Any member of a group knows all the servers that are responsible for signature aggregation of his group. When a member
signs a block certificate, he sends his signature to \textbf{all} the severs that aggregate the signatures of his group.
These servers aggregate the signatures they receive and then send the aggregated signature to the delegates.
Furthermore, the delegates aggregate these signatures to produce the final block certificate
and then include it in the next block.

The role of the delegates in the signature aggregation is limited. The important part of the work is done by ZK-EDB
servers. As long as there are enough honest ZK-EDB servers, the network will be able to perform signature aggregation
even if the delegates are malicious. (See Section~\ref{subsec:recovery})

\subsection{Estimating A User's Stake}\label{subsec:user's-stake}

In a proof of stake system the influence of a user in the consensus protocol should be proportional to the amount
of stake the user has in the system. Conventionally in these systems, a user's stake is considered to be equal with the
amount of native system tokens, the user has "staked" in the system. A user stakes his tokens by locking them in
his account or a staking account for some period of time, and during this time he will not be able to transfer
his tokens.

Unfortunately, there is a subtle problem with this approach. It is not clear that in a real world economic system
how much of the main currency of the system can be locked and kept out of the circulation indefinitely. It seems that
this amount for currencies like US dollar, is quite low comparing to the total market cap of the currency.
This means that for a real world currency this type of staking mechanisms will result in putting the
fate of the system in the hands of the owners of a small fraction of the total supply of a currency.

To mitigate this problem, Argennon uses a hybrid approach for estimating the stake of a user.
Every \texttt{stakingDuration} blocks, which is called a \emph{staking period}, Argennon calculates
a \emph{trust value} for each user. The user's stake
at time step \(t\), is estimated based on the user's trust value and his ARG balance:
\begin{equation}
    S_{u,t} = \min (B_{u,t}, Trust_{u,k})\ ,\label{eq:stake}
\end{equation}
where:
\begin{itemize}
    \item \(S_{u,t}\) is the stake of user \(u\) at time step \(t\).
    \item \(B_{u,t}\) is the ARG balance of user \(u\) at time step \(t\).
    \item \(Trust_{u,k}\) is an estimated trust value for user \(u\) at staking period \(k\).
\end{itemize}

Argennon users can lock their ARG tokens in their account for any period of time. During this time a user
will not be able to transfer his tokens and there is no way for cancelling a lock.
The trust value of a user is calculated based on the amount of his locked tokens and the
Exponential Moving Average (EMA) of his ARG balance:
\begin{equation}
    Trust_{u,k} = L_{u,k} + M_{u,t_k}\ ,\label{eq:trust}
\end{equation}
where
\begin{itemize}
    \item $L_{u,k}$ is the amount of locked tokens of user $u$, whose release time is \textbf{after the end} of
    the staking period $k+1$.
    \item $M_{u,t_k}$ is the Exponential Moving Average (EMA) of the ARG balance of user \(u\) at time step \(t_k\).
    $t_k$ is the start time of the staking period $k$.
\end{itemize}

In Argennon a user who held ARGs and participated in the consensus for a long time is more trusted
than a user with a higher balance whose balance has increased recently. An attacker who has obtained a large
amount of ARGs, also needs to hold them for a long period of time before being able to attack the system.

For calculating the EMA of a user's balance at time step \(t\), we can use the following
recursive formula:
\[
    M_{u,t} = (1 - \alpha) M_{u,t-1} + \alpha B_{u,t} = M_{u,t-1} + \alpha (B_{u,t} - M_{u,t-1})\ ,
\]
where the coefficient \(\alpha\) is a constant smoothing factor between \(0\) and \(1\), which represents the
degree of weighting decrease. A higher \(\alpha\) discounts older observations faster.

Usually an account balance will not change in every time step, and we can use older values of EMA for calculating
\(M_{u,t}\): (In the following equations the \(u\) subscript is dropped for simplicity)
\[
    M_{t} = (1 - \alpha)^{t-k}M_{k} + [1 - (1 - \alpha)^{t - k}]B\ ,
\]
where:
\[
    B = B_{k+1} = B_{k+2} = \dots = B_{t}\ .
\]
We know that when \(|nx| \ll 1\) we can use the binomial approximation \({(1 + x)^n \approx 1 + nx}\). So, we can
further simplify this formula:
\[
    M_{t} = M_{k} + (t - k) \alpha (B - M_{k})\ .
\]

For choosing the value of \(\alpha\) we can consider the number of time steps that the trust value of a user needs
for reaching a specified fraction of his account balance. We know that for large \(n\) and \(|x| < 1\) we have
\((1 + x)^n \approx e^{nx}\), so by letting \(M_{u,k} = 0\) and \(n = t - k\) we can write:
\begin{equation}
    \alpha =- \frac{\ln\left(1 - \frac{M_{n+k}}{B}\right)}{n}\ .\label{eq:alpha}
\end{equation}
The value of \(\alpha\) for a desired configuration can be calculated by this equation. For instance, we could
calculate the \(\alpha\) for a relatively good configuration in which \(M_{n+k} = 0.8B\) and \(n\) equals to the
number of time steps of 10 years.


\subsection{The Recovery Protocol}\label{subsec:recovery}
\note{not yet written...}

\subsection{Analysis}\label{subsec:consensus-math}
\note{not yet written...}
