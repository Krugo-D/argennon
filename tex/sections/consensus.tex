%! Author = aybehrouz

The credibility of a block of the Argennon blockchain is determined based on the certificates it has received
from different sets of users, knows as committees. There are two primary types of certificate committees in
Argennon: the committee of \emph{delegates} and the committee
of \emph{validators}. Argennon has one committee of delegates and $m$ committees of validators.

The committee of delegates issues a certificate for every block of the Argennon blockchain, and each
committee of validators issues a certificate every $m$ blocks. A validators' committee will
certify a block only if it has already been certified by the committee of delegates. Every committee of validators has
an index between $0$ and $m - 1$, and it issues a certificate for block number $n$, if $n$ modulo $m$ equals
the committee index.

Every block of the Argennon blockchain needs a certificate from both the committee of delegates and
the committee of validators. \textbf{A block is considered final after its next block receives both of
its certificates.}

In addition to primary committees, Argennon has several community driven committees. Certificates of these
committees are not required for block finality, but they could be used by members of the
validators' committee to better decide about the validity of a block.

When an anomaly is detected in the consensus mechanism, the \emph{recovery} protocol is initiated by nodes. The
recovery protocol is designed to be resilient to many types of attacks in order to be able to restore the normal
functionality of the system.

\subsection{The Committee of Delegates}\label{subsec:the-committee-of-delegates}

The committee of delegates is a small committee of trusted delegates, elected by Argennon users through the
Argennon Decentralized Autonomous Governance system (ADAGs\footnote{pronounced \textipa{/eI-dagz/}.}).
At the start of the Argennon mainnet, this committee will have
five members, and later its size could be changed by the ADAGs in a procedure described
in Section~\ref{sec:adags}.

The committee of delegates is responsible for creating new blocks of the Argennon blockchain, and it issues a
certificate for every block of the Argennon blockchain. A certificate needs to be signed
by \textbf{all} of the committee members in order to be considered valid.

Besides the main delegates' committee, a reserve committee of delegates consists of three members is elected by users
either through the ADAGs or by \emph{emergency agreement} during the recovery protocol. In case the main committee
fails to generate new blocks or behaves maliciously, the task of
block generation will be assigned to the reserve committee until a new main delegates' committee is elected
through the ADAGs.

Usually, the delegates are large organizations, and they have enough computational resources to generate blocks
very fast. However, a block is not completely final if it does not have the certificate of the validators.
A certified block by the delegates will not be accepted by the network, if the last block certified by
the validators is behind it more than a certain number of blocks.

The committee of delegates may use any type of agreement protocol to reach consensus on the
next block. Usually a very simple and fast protocol can do the job: one of the members
is randomly chosen as the proposer, and other members vote "yes" or "no" on the proposed block.

If one of the delegates lose its network connectivity, no new blocks can be generated. For this reason,
the delegates should invest on different types of communication infrastructure, to make sure they never lose
connectivity to each other and to the Argennon network.

\subsection{Validators}\label{subsec:validators-committee}

The Argennon protocol calculates a stake value for every account, which is an estimate of a user's stake in the
system, and is measured in ARGs. Any account whose stake value is higher than
\texttt{minValidatorsStake} threshold is considered a \emph{validator}.
The \texttt{minValidatorsStake}
threshold is determined by the ADAGs, but it can never be higher than $500$ ARGs.

Every \texttt{committeeLifeTime} blocks, randomly $m$ committees are selected from
validators, in a way that the total stake of members of each committee is approximately equal, and every
account is a member of \textbf{at least} one committee.

Every validator has a status which can be either \texttt{online} or \texttt{offline}.
This status is stored in the ARG smart contract and is a part of the staking database. A validator can change
his status through a method invocation
from the ARG smart contract. When an account sets its status to \texttt{offline}, it receives a small reward, and
it can not change it back to \texttt{online} for \texttt{statusCoolDown} number of blocks.

\note{When a validator changes his status, the change has no effect until the block containing the status
change transaction gets certified by his committee.}

A block certificate issued by some members of a validators' committee is considered valid, if according to
the staking database of the previous block \textbf{certified by the same committee}, we have:\footnote{If we calculate
the stake values based on the previous block a malicious committee can affect the validators of the next block.}
\begin{itemize}
    \item The total stake of \texttt{online} members of the committee is higher than \texttt{minOnlineStake} percent
    of the total stake of the committee. This threshold can be changed by the ADAGs, but it can never be lower
    than $2/3$.
    \item All signers of the certificate have \texttt{online} status.
    \item The sum of stake values of certificate signers is higher than 80 percent of the total stake
    of the committee members that have \texttt{online} status.
\end{itemize}

The delegates can generate blocks very fast. Therefore, the Argennon blockchain always has an
unvalidated part which contains blocks that have a certificate from the committee of delegates but have not received
a certificate from the validators yet.

As we mentioned before, the block with height $n$ needs a certificate from the committee of
validators with index $n$ modulo $m$. To decide about signing a block certificate, a validator
checks the conditional validity\footnote{See Section~\ref{subsec:block-validation}} of the block, and
if the block is valid he issues
an "accept" signature. If the block is invalid, he initiates the recovery protocol. The validator will broadcast the
certificate \textbf{only after} he sees the certificate of the validators of the previous block.

Some validators may also require seeing a certificate from
some community driven committee for the previous block, before they broadcast an "accept" certificate for a block.

So in Argennon, the block validation by committees is performed in parallel, and validators
do not wait for seeing the validators' certificate of the previous block to start transaction validation. On the
other hand, the block certificates are published and broadcast sequentially. This ensures that an invalid
fork made by malicious delegates will not receive any certificates from validators.

The value of $m$ is determined by the ADAGs. but it can never be higher than $25$. This way, it is guaranteed
that on average, any block of the Argennon blockchain is validated by at least $0.02$ of the total ARG supply.

block certificates issued by committees of validators are included in the blocks of the Argennon blockchain.
A block can contain multiple certificates, provided that
those certificates belong to consecutive blocks.

\subsection{Status Blocks}\label{subsec:status-blocks}

If according to the staking database of block $n$, the total online stake of the committee with index $n$ modulo $m$ is
lower than \texttt{minOnlineStake} threshold, the block $n + m$ can never be certified by validators.

To prevent blockchain from halting in such situations, the protocol performs a predefined partial
reshuffling of committee members.
In this reshuffling which is based on the block random seed, some online members from other committees will
be moved to the committee without enough online stake to make it active again.

If the reshuffling can not solve the problem due to low total online stake, the protocol requires the next block
of the blockchain to be a special \emph{status} block. A status block is a special block which can only contain
status change transactions. The status block need to be certified by the delegates and by 2/3 of the total stake of
the validators. The \texttt{online}/\texttt{offline} status of validators will not be considered in the validity of
the status block certificate.\footnote{Theoretically at the status block, the total online stake of the system could
be very low, even zero. Therefore, the status block should not be certified only by online stake.} After applying
the transactions of the status block, the total online stake of validators
must go higher than \texttt{minOnlineStake} threshold.

\subsection{Signature Aggregation}\label{subsec:sig-agg}

In Argennon, signature aggregation is mostly performed by ZK-EDB servers. To distribute the aggregation workload
between different servers, Every committee of validators is divided into pre-determined groups, and each ZK-EDB
server is responsible for signature aggregation of one group. To make sure that there is enough redundancy, the
total number of groups should be less than the number of ZK-EDB servers and each group should be assigned to
multiple ZK-EDB servers.

Any member of a group knows all the servers that are responsible for signature aggregation of his group. When a member
signs a block certificate, he sends his signature to \textbf{all} the severs that aggregate the signatures of his group.
These servers aggregate the signatures they receive and then send the aggregated signature to the delegates.
Furthermore, the delegates aggregate these signatures to produce the final block certificate
and then include it in the next block.

The role of the delegates in the signature aggregation is limited. The important part of the work is done by ZK-EDB
servers. As long as there are enough honest ZK-EDB servers, the network will be able to perform signature aggregation
even if the delegates are malicious.

\subsection{The Recovery Protocol}\label{subsec:recovery}

The recovery protocol is a resilient protocol intended for recovering the Argennon blockchain from critical situations.
In the terminology of the CAP theorem, the recovery protocol is designed to choose consistency over availability
and is not a protocol supposed to be executed occasionally. Ideally this protocol should never be used
during the lifetime of the Argennon blockchain.

\subsection{Estimating A User's Stake}\label{subsec:user's-stake}

In a proof of stake system the influence of a user in the consensus protocol should be proportional to the amount
of stake the user has in the system. Conventionally in these systems, a user's stake is considered to be equal with the
amount of native system tokens, the user has "staked" in the system. A user stakes his tokens by locking them in
his account or a staking account for some period of time, and during this time he will not be able to transfer
his tokens.

Unfortunately, there is a subtle problem with this approach. It is not clear that in a real world economic system
how much of the main currency of the system can be locked and kept out of the circulation indefinitely. It seems that
this amount for currencies like US dollar, is quite low comparing to the total market cap of the currency.
This means that for a real world currency this type of staking mechanisms will result in putting the
fate of the system in the hands of the owners of a small fraction of the total supply of a currency.

To mitigate this problem, Argennon uses a hybrid approach for estimating the stake of a user.
Every \texttt{stakingDuration} blocks, which is called a \emph{staking period}, Argennon calculates
a \emph{trust value} for each user. The user's stake
at time step \(t\), is estimated based on the user's trust value and his ARG balance:
\begin{equation}
    S_{u,t} = \min (B_{u,t}, Trust_{u,k})\ ,\label{eq:stake}
\end{equation}
where:
\begin{itemize}
    \item \(S_{u,t}\) is the stake of user \(u\) at time step \(t\).
    \item \(B_{u,t}\) is the ARG balance of user \(u\) at time step \(t\).
    \item \(Trust_{u,k}\) is an estimated trust value for user \(u\) at staking period \(k\).
\end{itemize}

Argennon users can lock their ARG tokens in their account for any period of time. During this time a user
will not be able to transfer his tokens and there is no way for cancelling a lock.
The trust value of a user is calculated based on the amount of his locked tokens and the
Exponential Moving Average (EMA) of his ARG balance:
\begin{equation}
    Trust_{u,k} = L_{u,k} + M_{u,t_k}\ ,\label{eq:trust}
\end{equation}
where
\begin{itemize}
    \item $L_{u,k}$ is the amount of locked tokens of user $u$, whose release time is \textbf{after the end} of
    the staking period $k+1$.
    \item $M_{u,t_k}$ is the Exponential Moving Average (EMA) of the ARG balance of user \(u\) at time step \(t_k\).
    $t_k$ is the start time of the staking period $k$.
\end{itemize}

In Argennon a user who held ARGs and participated in the consensus for a long time is more trusted
than a user with a higher balance whose balance has increased recently. An attacker who has obtained a large
amount of ARGs, also needs to hold them for a long period of time before being able to attack the system.

For calculating the EMA of a user's balance at time step \(t\), we can use the following
recursive formula:
\[
    M_{u,t} = (1 - \alpha) M_{u,t-1} + \alpha B_{u,t} = M_{u,t-1} + \alpha (B_{u,t} - M_{u,t-1})\ ,
\]
where the coefficient \(\alpha\) is a constant smoothing factor between \(0\) and \(1\), which represents the
degree of weighting decrease. A higher \(\alpha\) discounts older observations faster.

Usually an account balance will not change in every time step, and we can use older values of EMA for calculating
\(M_{u,t}\): (In the following equations the \(u\) subscript is dropped for simplicity)
\[
    M_{t} = (1 - \alpha)^{t-k}M_{k} + [1 - (1 - \alpha)^{t - k}]B\ ,
\]
where:
\[
    B = B_{k+1} = B_{k+2} = \dots = B_{t}\ .
\]
We know that when \(|nx| \ll 1\) we can use the binomial approximation \({(1 + x)^n \approx 1 + nx}\). So, we can
further simplify this formula:
\[
    M_{t} = M_{k} + (t - k) \alpha (B - M_{k})\ .
\]

For choosing the value of \(\alpha\) we can consider the number of time steps that the trust value of a user needs
for reaching a specified fraction of his account balance. We know that for large \(n\) and \(|x| < 1\) we have
\((1 + x)^n \approx e^{nx}\), so by letting \(M_{u,k} = 0\) and \(n = t - k\) we can write:
\begin{equation}
    \alpha =- \frac{\ln\left(1 - \frac{M_{n+k}}{B}\right)}{n}\ .\label{eq:alpha}
\end{equation}
The value of \(\alpha\) for a desired configuration can be calculated by this equation. For instance, we could
calculate the \(\alpha\) for a relatively good configuration in which \(M_{n+k} = 0.8B\) and \(n\) equals to the
number of time steps of 10 years.

\subsection{Analysis}\label{subsec:consensus-math}
\note{not yet written...}
