%! Author = aybehrouz

The Argennon Smart Contract Execution Environment has two persistent memory areas: \emph{code area}, and \emph{heap}.
Code area stores the Argennon Standard Representation of
applications\footnote{also it stores applications' constants.}, and heap stores heap chunks.
Both of these data elements, ASRs and heap chunks, can be considered as continuous arrays of bytes.
Throughout this chapter, we shall call these data elements \emph{objects}.


\section{Storage Pages}\label{sec:storage-pages}

In the AscEE persistence layer, similar objects are clustered together and constitute a bigger data element which we
call
a \emph{page}.\footnote{we avoid calling them clusters, because usually a cluster refers to a \emph{set}. AscEE object
clusters are not sets. They are ordered lists, like a page containing an ordered list of words or sentences.}
A page is an ordered list of an arbitrary number of objects, which their order reflects the order they were added to
the page:
\[
    P = (O_1,O_2,\dots,O_n),\quad i < j \; \Leftrightarrow \; \textrm{$O_i$ was added before $O_j$}\ \cdot
\]

A page of the AscEE storage should contain objects that have very similar access patterns. Ideally, when a page is needed
for validating a block, almost all of its objects should be needed for either reading or writing. We prefer
that the objects are needed for the same access type. In other words, the objects of a page are chosen in a way that
for validating a block, we usually need to either read all of them or modify\footnote{and probably read.} all of them.


\section{Publicly Verifiable Database Servers}\label{sec:zk-edb}

Pages of the AscEE storage are persisted using \emph{dynamic cryptographic accumulators}. Argennon
has three dynamic cryptographic accumulators: \emph{staking} database, which stores all the data that is associated with
the Argennon consensus protocol, \emph{code} database, which stores the AscEE code area, and \emph{heap} database,
which stores the AscEE heap. The commitment of these three accumulators are included in every block of the Argennon
blockchain. These accumulators, in the Argennon network, are hosted on special database nodes called Publicly
Verifiable Database (PV-DB) servers.
We consider the following properties for a PV-DB:
\begin{itemize}
    \item The PV-DB contains a mapping from a set of keys to a set of values.
    \item Every state of the database has a commitment \(C\).
    \item The PV-DB has a method \((D, \pi) = \text{get}(x)\), where \(x\) is a key and \(D\) is the associated data
    with \(x\), and \(\pi\) is a proof.
    \item A user having \(C\) and \(\pi\) can verify that \(D\) is really associated with \(x\), and \(D\) is not
    altered. Consequently, a user who can obtain \(C\) from a trusted source does not need to trust the PV-DB\@.
    \item Having \(\pi\) and \(C\) a user can compute the commitment \(C'\) for the database in which \(D'\) is
    associated with \(x\) instead of \(D\).
\end{itemize}

Pages of the AscEE storage are stored in the PV-DBs, with an index: \texttt{pageIndex} as their key.
The \texttt{pageIndex} is required to be smaller than a certain value, determined by the
protocol, to facilitate the usage of PV-DBs that are based on vector commitments.
For this reason, the AscEE clustering algorithm always tries to reuse indices and keep the number of used indices
as low as possible.

The commitments of the AscEE cryptographic accumulators are affected by the way data objects are clustered. Therefore,
the Argennon clustering algorithm has to be a part of the consensus protocol.

Every block of the Argennon blockchain contains a set of \emph{clustering directives}. These directives
can only modify pages that were used for validating the block, and can
include directives for moving an object from one page to another or directives specifying which pages will contain
the newly created objects. These directives are always executed by nodes at the end of block validation.

A block proposer could obtain clustering directives from any third party source\footnote{we can say the AscEE clustering
algorithm is essentially off-chain.}. This will not
affect Argennon security, since the integrity of a database can not be altered by clustering directives.
Those directives can only affect the performance of the Argennon network, and directives of a single block can
not affect the performance considerably.

\subsection{Vector Commitments}\label{subsec:impl-zk-edbs}

Informally, vector commitments allow committing to an ordered sequence of $q$ values
(i.e.\ a vector), rather than to single messages. This is done in a way such that it is later possible
to open the commitment with respect to specific positions (e.g., to prove that $m_i$ is the $i$-th committed
message). More precisely, vector commitments are required to satisfy what is called position binding.
Position binding states that an adversary should not be able to open a commitment to two different
values at the same position. While this property, by itself, would be trivial to realize using standard
commitment schemes, what makes vector commitments interesting is that they are concise, i.e.,
the size of the commitment string as well as the size of each opening \textbf{is independent of the
vector length}.

\note{not yet written...}


\section{Object Clustering Algorithm}\label{sec:clustering}

\note{not yet written...}


