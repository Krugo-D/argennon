%! Author = aybehrouz


The Argennon Virtual Machine has two persistent memory areas: \emph{method area}, and \emph{heap}. Method area stores
method byte codes \footnote{also it stores constant area blocks.}, and heap stores memory chunks. Both of these
data elements, methods and chunks, can be considered as continuous pieces of byte addressable memory
which we shall call \emph{objects} throughout this chapter.

In the AVM persistence layer, similar objects are clustered together and constitute a bigger data element which we call a
\emph{page}.\footnote{we avoided calling them a cluster, because usually a cluster refers to a \emph{set}. AVM object
clusters are not sets. They are ordered lists, like a page containing an ordered list of words or sentences.}
A page is an ordered list of an arbitrary number of objects, which their order reflects the order they were added to
the page:
\[
    P = (O_1,O_2,\dots,O_n),\quad i < j \; \Leftrightarrow \; \textrm{$O_i$ was added before $O_j$}
\]

A page of the AVM storage should contain objects that have very similar access pattern. We expect that when a page
is needed for validating a block, almost all of its objects are needed. We also prefer that the objects of the page
are needed for the same access type. In other words, the objects of a page are chosen in a way that
for validating a block, we usually need to either read all of them or modify\footnote{and probably read.} all of them.

Pages of the AVM storage space are persisted using updatable zero-knowledge elementary databases (ZK-EDB). Argennon
has three zero-knowledge databases. \emph{Staking} database which stores all the data that is associated with
the Argennon agreement protocol. \emph{Method} database which stores the AVM method area, and \emph{Heap} database which
stores the AVM heap. The commitment of these three ZK-EDBs are included in every block of the Argennon blockchain.

We consider the following properties for a ZK-EDB:
\begin{itemize}
    \item The ZK-EDB contains a mapping from a set of keys to a set of values.
    \item Every state of the database has a commitment \(C\).
    \item The ZK-EDB has a method \((D, \pi) = get(x)\), where \(x\) is a key and \(D\) is the associated data
    with \(x\), and \(\pi\) is a proof.
    \item A user can use \(C\) and \(\pi\) to verify that \(D\) is really associated with \(x\), and \(D\) is not
    altered. Consequently, a user who can obtain \(C\) from a trusted source does not need to trust the ZK-EDB\@.
    \item Having \(\pi\) and \(C\) a user can compute the commitment \(C'\) for the database in which \(D'\) is
    associated with \(x\) instead of \(D\).
\end{itemize}

Every page of the AVM storage is stored with an index as its key: \texttt{pageIndex}. The \texttt{pageIndex} is
required to be smaller than a certain value determined by the
protocol.\footnote{this facilitates the usage of ZK-EDBs that use vector commitments.}
Therefore, the AVM clustering algorithm tries to reuse indices and keep the number of used indices as low as
possible.

In Argennon \emph{light} nodes do not keep a full copy of the AVM storage and for emulating the Argennon Virtual Machine
( i.e.~validating transactions) they need to connect to a ZK-EDB and retrieve the required pages.
Because light node need to be able to validate block certificates, they usually cache a large part of the staking
database, and keep their cache updated to make sure they can keep themselves in sync with the blockchain.

The commitments of the AVM databases are dependant to the clustering of data objects. So, the clustering algorithm of
the AVM storage has to be a part of the Argennon agreement protocol. Clustering of objects is one of the
tasks that a block proposer performs in Argennon.

Every block of the Argennon blockchain contains a set of \emph{clustering directives}. These directives
can only modify pages that were used for validating the block, and can
include directives for moving an object from one page to another or directives specifying which pages will contain
the newly created objects. These directives will always be run by nodes at the end of block validation.

A block proposer usually obtains clustering directives from a third party source who enjoys a considerable amount of
computational power. This federated approach does not affect Argennon security, because the integrity of a
database can not be altered by clustering directives. Those
directives can only affect the performance of the argennon blockchain, and directives of a single block can
not change the performance much.
