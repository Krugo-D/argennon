%! Author = aybehrouz

The Argennon Virtual Machine has two persistent memory areas: \emph{method area}, and \emph{heap}. Method area stores
bytecodes of methods\footnote{also it stores constant area blocks.}, and heap stores memory chunks. Both of these
data elements, bytecodes and chunks, can be considered as continuous pieces of byte addressable memory. Throughout this
chapter, we shall call these data elements \emph{objects}.


\section{Storage Pages}\label{sec:storage-pages}

In the AVM persistence layer, similar objects are clustered together and constitute a bigger data element which we call
a \emph{page}.\footnote{we avoid calling them clusters, because usually a cluster refers to a \emph{set}. AVM object
clusters are not sets. They are ordered lists, like a page containing an ordered list of words or sentences.}
A page is an ordered list of an arbitrary number of objects, which their order reflects the order they were added to
the page:
\[
    P = (O_1,O_2,\dots,O_n),\quad i < j \; \Leftrightarrow \; \textrm{$O_i$ was added before $O_j$}\ .
\]

A page of the AVM storage should contain objects that have very similar access pattern. We expect that when a page
is needed for validating a block, almost all of its objects are needed for either reading or writing. We also prefer
that the objects are needed for the same access type. In other words, the objects of a page are chosen in a way that
for validating a block, we usually need to either read all of them or modify\footnote{and probably read.} all of them.

\section{Zero-knowledge Databases}\label{sec:zk-edb}

Pages of the AVM storage are persisted using updatable zero-knowledge elementary databases (ZK-EDB). Argennon
has three zero-knowledge databases: \emph{staking} database, which stores all the data that is associated with
the Argennon consensus protocol. \emph{method} database, which stores the AVM method area, and \emph{heap} database,
which stores the AVM heap. The commitment of these three ZK-EDBs are included in every block of the Argennon blockchain.

We consider the following properties for a ZK-EDB:
\begin{itemize}
    \item The ZK-EDB contains a mapping from a set of keys to a set of values.
    \item Every state of the database has a commitment \(C\).
    \item The ZK-EDB has a method \((D, \pi) = \text{get}(x)\), where \(x\) is a key and \(D\) is the associated data
    with \(x\), and \(\pi\) is a proof.
    \item A user having \(C\) and \(\pi\) can verify that \(D\) is really associated with \(x\), and \(D\) is not
    altered. Consequently, a user who can obtain \(C\) from a trusted source does not need to trust the ZK-EDB\@.
    \item Having \(\pi\) and \(C\) a user can compute the commitment \(C'\) for the database in which \(D'\) is
    associated with \(x\) instead of \(D\).
\end{itemize}

Pages of the AVM storage are stored in the ZK-EDBs, with an index: \texttt{pageIndex} as their key.
The \texttt{pageIndex} is required to be smaller than a certain value, determined by the
protocol, to facilitate the usage of ZK-EDBs that are based on vector commitments.
For this reason, the AVM clustering algorithm always tries to reuse indices and keep the number of used indices
as low as possible.

The commitments of the AVM ZK-EDBs are affected by the way data objects are clustered. Therefore,
the Argennon clustering algorithm has to be a part of the consensus protocol.

Every block of the Argennon blockchain contains a set of \emph{clustering directives}. These directives
can only modify pages that were used for validating the block, and can
include directives for moving an object from one page to another or directives specifying which pages will contain
the newly created objects. These directives are always executed by nodes at the end of block validation.

A block proposer could obtain clustering directives from any third party source\footnote{we can say the AVM clustering
algorithm is essentially off-chain.}. This will not
affect Argennon security, since the integrity of a database can not be altered by clustering directives.
Those directives can only affect the performance of the Argennon network, and directives of a single block can
not affect the performance considerably.

\subsection{Vector Commitments}\label{subsec:impl-zk-edbs}

\note{not yet written...}

\section{Object Clustering Algorithm}\label{sec:clustering}

\note{not yet written...}


